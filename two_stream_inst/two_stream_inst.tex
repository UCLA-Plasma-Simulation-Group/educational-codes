\documentclass[12pt]{article}
\usepackage{graphicx}
\usepackage{float}
\usepackage{caption}
\usepackage{subcaption}
\usepackage{fullpage}
\usepackage{lastpage}
\usepackage{fancyhdr}
\usepackage{wrapfig}
\usepackage{lipsum}
\usepackage{mathtools}
\usepackage{amsfonts}
\usepackage{enumitem}
\usepackage{amsmath}
\usepackage{listings}
\usepackage[margin=0.8in]{geometry}

\newcommand{\s}{\hspace{5pt}}
\newcommand{\half}{\frac{1}{2}}
\newcommand{\R}{\mathbb{R}}
\newcommand{\C}{\mathbb{C}}
\newcommand{\p}{\partial}
\newcommand{\mb}{\mathbf}

\newenvironment{m}{\begin{pmatrix}}{\end{pmatrix}}
\newenvironment{e}{\begin{enumerate}[label=(\alph*)]}{\end{enumerate}}

\setlength{\parindent}{0in}

\begin{document}

\title{\vspace{-5ex}Plasma Educational Notes: Two-Stream Instability\vspace{-1ex}}
\date{\vspace{-1ex}\today}
\author{Kyle Miller \& Lance Hildebrand}
\maketitle

\section*{Introduction}

A two-stream instability occurs when two species (either the same or different) in a plasma have different drift velocities. Depending on the physical parameters, modes can then arise which are unstable and grow exponentially. To begin, we derive the general dispersion relation for these instabilities.

\section*{General Dispersion Relation}

Consider two cold species, which we will generically label species 1 and species 2, each with constant drift velocity $\mb{v}_{0,1}$ and $\mb{v}_{0,2}$ and fluctuating velocity $\tilde{\mb{v}}_1$ and $\tilde{\mb{v}}_2$, respectively. The linearized Navier-Stokes equation for each species is then

\begin{align*}
\frac{d}{d t} \tilde{\mb{v}}_s &= \frac{\p}{\p t} \tilde{\mb{v}}_s + \mb{v}_{0,s} \cdot \nabla \tilde{\mb{v}}_s = \frac{q_s}{m_s} \tilde{\mb{E}}.
\end{align*}

In addition, the continuity equation for each species is

\begin{align*}
\frac{\p}{\p t} \tilde{n}_s + n_{0,s} \nabla \cdot \tilde{\mb{v}}_s + \mb{v}_{0,s} \cdot \nabla \tilde{n}_s &= 0.
\end{align*}

Poisson's equation then yields

\begin{align*}
\nabla \cdot \tilde{\mb{E}} = 4\pi \sum_{s}q_s \tilde{n}_s.
\end{align*}

If we assume a plane wave solution of the form $\tilde{\mb{E}} = \mb{E}_0 e^{i(\mb{k} \cdot \mb{r} - \omega t)}$, then the dynamical equation turns into

\begin{align*}
(-i\omega + i \mb{k} \cdot \mb{v}_{0,s})\tilde{\mb{v}}_s = \frac{q_s}{m_s} \tilde{\mb{E}} \\
\Rightarrow \tilde{\mb{v}}_s = \frac{q_s \tilde{\mb{E}}}{i m_s (-i\omega + i \mb{k} \cdot \mb{v}_{0,s})}.
\end{align*}

Similarly, the continuity equation can be rewritten as

\begin{align*}
-i\omega \tilde{n}_s + i n_{0,s} \mb{k} \cdot \tilde{\mb{v}}_s + i \mb{k} \cdot \mb{v}_{0,s} \tilde{n}_s = 0 \\
\Rightarrow \tilde{n}_s = \frac{n_{0,s} \mb{k} \cdot \tilde{\mb{v}}_s}{\omega - \mb{k} \cdot \mb{v}_{0,s}} = \frac{-q_s n_{0,s} \mb{k} \cdot \tilde{\mb{E}}}{i m_s (\omega - \mb{k} \cdot \mb{v}_{0,s})^2}.
\end{align*}

If we substitute this expression for $\tilde{n}_s$ into Poisson's equation, after rearranging we find that

\begin{align*}
\left(1 - \sum_s\frac{\omega_{p,s}^2}{(\omega - \mb{k} \cdot \mb{v}_{0,s})^2}\right)i \mb{k} \cdot \tilde{\mb{E}} = 0.
\end{align*}

Recognizing that $\nabla \cdot \tilde{\mb{D}} = \nabla \cdot (\epsilon \tilde{\mb{E}}) = \epsilon \mb{k} \cdot \tilde{\mb{E}}$, the term in parenthesis is then our dielectric constant. Setting this equal to zero gives the dispersion relation as

\begin{align*}
1 - \frac{\omega_{p,1}^2}{(\omega - \mb{k} \cdot \mb{v}_{0,1})^2} - \frac{\omega_{p,2}^2}{(\omega - \mb{k} \cdot \mb{v}_{0,2})^2} = 0.
\end{align*}

This equation can be used for various types of two-stream instabilities, for which the parameters $\omega_{p,s}$ and $\mb{v}_{0,s}$ can be adjusted. Following are several specific cases of the two-stream instability.

\section*{Farley-Buneman Instability}
Consider a stationary background of ions with the electron plasma moving with a constant drift velocity $\mb{v}_0$. This could be produced, for example, by a current-carrying plasma. Then the dispersion relation is reduced to

\begin{align*}
1 = \omega_{pe}^2 \left(\frac{m_e/m_i}{\omega^2} + \frac{1}{(\omega - \mb{k} \cdot \mb{v}_0)^2}\right).
\end{align*}



\section*{Weak Cold Beam Instability}
Consider a stationary electron-ion plasma with a fast, weak beam of electrons passing through it. Here ``fast'' implies $v_b \gg \bar{v}_e, \bar{v}_i$, ``weak'' implies $n_b/n_0 \ll 1$, and ``cold'' implies $v_b \gg \bar{v}_b$. Since $\omega_{pi} \ll \omega_{pe}$, we neglect the ion contribution to the dispersion relation and obtain

\begin{align*}
1 = \frac{\omega_{pe}^2}{\omega^2} + \frac{\omega_{pb}^2}{(\omega-\mb{k} \cdot \mb{v}_b)^2}.
\end{align*}

\section*{References}
[1] N. A. Krall and A. W. Trivelpiece, ``Principle of Plasma Physics,'' McGraw Hill, New York, 1973.

\end{document}
